\newpage
\thispagestyle{empty}
\mbox{}

\chapter{Estimación del gradiente}
\label{ch:chapter3}

Acá la idea es describir el algoritmo con su desarrollo matemático para que no salga nada de la nada (mas o menos 10-12 paginas en este capitulo).


Me quede aca, desde aca para abajo es sucio.

En donde $r_{i}=$ es la posición del robot i, ${\boldsymbol{\phi }}_{\boldsymbol{i}}\boldsymbol{=}\frac{\boldsymbol{2}\cdot\boldsymbol{\pi }\cdot\boldsymbol{i}}{\boldsymbol{N}}\boldsymbol{\ }$es el ángulo de rotación, $R_{\phi }$ \underbar{es la matriz de rotación} definida como \textbf{ }$\left[ \begin{array}{cc} {\boldsymbol{c}}_{\boldsymbol{\phi }} & \boldsymbol{-}{\boldsymbol{s}}_{\boldsymbol{\phi }} \\  {\boldsymbol{s}}_{\boldsymbol{\phi }} & {\boldsymbol{c}}_{\boldsymbol{\phi }} \end{array} \right]$\textbf{, }finalmente $e\ =\ {\left[1,0\right]}^T$, por simplicidad no se considera la dinámica de los robots.



\noindent La señal está definida según una función cuadrática $\boldsymbol{\sigma }\boldsymbol{(}\boldsymbol{r}\boldsymbol{)=}{\boldsymbol{r}}^{\boldsymbol{T}}\boldsymbol{\cdot}\boldsymbol{S}\boldsymbol{\cdot}\boldsymbol{r}\boldsymbol{+}{\boldsymbol{p}}^{\boldsymbol{T}}\boldsymbol{\cdot}\boldsymbol{r}\boldsymbol{+}\boldsymbol{q}$ si se tiene una formación de más de 4 robots se asume que la estimación es el gradiente de la función.

\noindent 
\[{\phi }_i={\phi }_{o}+\frac{2\cdot\pi \cdot{i}}{N} \] 

Con ${\phi }_{o}\left(t\right)=w_{o}\cdot{t}$la formación propuesta es adecuada para robots que se mueven en formación circular como vehículos aéreos no tripulados de área.

Problema en cuestión:

Se puede poner de tres formas:

Primera forma:

\begin{equation*}
	\hat{\nabla}{f}\left(c\right):=\frac{2}{{D}^2\cdot{N}}\cdot\sum_{i=1}^{N}f(r_{i})\cdot(r_{i}-c)
\end{equation*}

Donde:

\begin{equation*}
	\hat{\nabla}{f}\left(c\right) = \nabla{f}\left(c\right) + \varphi\left(D,c\right)
\end{equation*}

Segunda forma:

\begin{equation*}
	\frac{2}{{D}^2\cdot{N}}\cdot\sum_{i=1}^{N}f(r_{i})\cdot(r_{i}-c)=\nabla{f}\left(c\right) + \varphi\left(D,c\right)
\end{equation*}


Tercera forma:

\begin{equation*}
	\frac{2}{{D}^2\cdot{N}}\cdot\sum_{i=1}^{N}f(r_{i})\cdot(r_{i}-c)=\underbrace{\nabla{f}\left(c\right) + \varphi\left(D,c\right)}_{:=\hat{\nabla}{f}\left(c\right)}
\end{equation*}


Se tiene una función $f\left(r\right)$, donde $r$ definida en 2 dimensiones, que el gradiente en el punto máximo es 0 ($\mathrm{\nabla }f\left(r^*\right)=0$), pero en el punto del campo escalar será distinto de 0 ($\left(\mathrm{\nabla }\sigma \left(r\right)\neq 0\right),$ obviamente se ha de dar con ``situaciones espaciales'' diferentes lugares ($\forall r\neq r^*$) y finalmente el hessiano estará definido negativamente dado que es un máximo local, es decir, $H_{\sigma (r^*)}<-a\cdot{I}_{p}$ (con a $\mathrm{>}$ 0 e $I_p$ es una matriz identidad perteneciente al espacio $R^{pxp}$.



Parrafos que me pueden servir en algun momento:


Se tienen dos casos ${\left[A\right]}_{\alpha }=A$ si $A\le -\alpha \cdot{I}_{p}$  sino seria  ${\left[A\right]}_{\alpha }=-I_p$, el primero de los casos se utilizará si el centro de formación c está muy alejado de la fuente así se evita que la matriz se defina como semipositiva y tienda a alejar a los robots del punto de interés, además, cuando dicho punto ``c'' está cerca de la fuente se asume entonces que $A<-\alpha \cdot{I}_{p}$
