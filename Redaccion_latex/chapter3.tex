\newpage
\thispagestyle{empty}
\mbox{}

\chapter{Estimación del gradiente}
\label{ch:chapter3}

En [referencia bibliográfica] se divide el problema en dos pasos, el primero, consiste en modelar a los agentes con dinámicas no lineales y se estabilizan a una formación deseada. En segundo lugar, un algoritmo distribuido que permite a los agentes estimar el gradiente de un campo escalar en el centro de dicha formación y conducirla hasta su lugar de origen.

A. Agentes ??? 

Se consideran un grupo de N vehículos idénticos modelados con cinemática de uniciclo sujetos a una simple restricción no holonómica, tal que la dinámica de los agentes se define como:

\begin{equation*}
	\begin{aligned}
		\dot{r}_i&=v_i\left[cos\theta_i\hspace{1mm}sin\theta_i\right]^T\\
		\dot{\theta}_i&=u_i
	\end{aligned}
\end{equation*}

En donde, $r_i\in\mathbb{R}^2$ definiendo la posición de cada agente i, $\theta_i$ es el angulo de cabecera/rumbo (no se como se traduce esto preguntar), por ultimo $u_i$ y $v_i$ son entradas de control. Se asume que cada vehículo conoce su posición absoluta con respecto al marco inercial (preguntar también como se traduce esto) y ademas, los agentes serán capaces de intercambiar sus angulos de cabecera/rumbo??.

Particularizando para una distribución uniforme a lo largo de un circulo con radio D, un ángulo de rotacion $\phi_o\left(t\right)=w_ot$ y el centro de la formación c. 

\begin{equation*}
	r_i = c + D\cdot{R}_{\theta_i}\cdot{e}\hspace{10mm}{i = 1,...,N}
\end{equation*}

En donde, $r_{i}$ es la posición del robot i con respecto al radio del circulo, ${\phi }_{i}=\phi_o\frac{2\cdot\pi\cdot{i}}{N}$ es el ángulo de rotación, $R_{\phi }$ es la matriz de rotación definida como $\left[ \begin{array}{cc} {c}_{\phi } & -{s}_{\phi } \\  {s}_{\phi } & {c}_{\phi } \end{array} \right]$ y  $e\ =\ {\left[1,0\right]}^T$.

B. Signal Strength.

Se destaca que cada uno de los agentes deberá tener la capacidad de medir la intensidad de la señal mediante un sensor. En términos matemáticos y como en capítulos anteriores se anticipo, la distribución de la señal es una función espacial bidimensional que representa un campo escalar con un máximo o mínimo definido justo en la posición donde dicha se localiza. Por lo tanto, en este problema se va a considerar que la señal es emitida por una única fuente de modo que su punto de inflexión en $z_*$ es el único máximo definido del campo escalar.

En el capitulo (preguntar como hacer referencia) sección se hablo de una función lipzchiana que a su vez están eran necesariamentes cuadráticas, el hecho de considerar este tipo de funciones se debe 



