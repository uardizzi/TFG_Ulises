% +--------------------------------------------------------------------+
% | Copyright Page
% +--------------------------------------------------------------------+

\newpage

\thispagestyle{empty}
\setlength{\parindent}{0cm}

{\bf \large Desarrollo de sistemas de control cooperativos para USVs en tareas de bioinspección}
\vspace{0.15cm}

{\bf \large Resumen}\\
Un problema que existe en la actualidad es la existencia de sustancias contaminantes de origen biológico en aguas continentales. En este Trabajo de Fin de Grado se hará uso  de un grupo de vehículos de superficie (USV), describiendo un enjambre robótico, para que monitoricen de modo autónomo grandes superficies marítimas. Dicha monitorización se hará mediante la detección y demarcación de las sustancias. Estas tareas son realizables mediante la acción conjunta de tres algoritmos: el primero de ellos estima el gradiente de máxima concentración de sustancia para el seguimiento y demarcación de zonas contaminadas en base a las medidas tomada por lo sensores de los vehículos, el segundo algoritmo es un control cooperativo que los coordina para que se dispongan uniformemente en una formación circular y el último permite la navegación en formación sobre la superficie marítima.

\vspace{0.15cm}


{\bf Palabras clave:} Enjambre robótico, sistemas multiagente, gradiente, formación circular, vehículos autónomos, contaminación en superficies marítimas
\vspace{0.15cm}

{\bf \large Abstract}\\
A problem that exists today is the existence of polluting substances of biological origin in continental waters. In this Final Degree Project, a group of surface vehicles (USV) will be used, describing a robotic swarm, to autonomously monitor large maritime surfaces. This monitoring will be done through the detection and demarcation of the substances. These tasks can be carried out through the joint action of three algorithms: the first one estimates the gradient of maximum substance concentration for the monitoring and demarcation of contaminated areas based on the measurements taken by the vehicle sensors, the second algorithm is a cooperative control that coordinates them so that they are uniformly arranged in a circular formation and the latter allows navigation in formation on the sea surface.

\vspace{0.15cm}

{\bf Key works:} swarm robotics, multi-agent systems, gradient, circle formation, autonomous vehicles, pollution on maritime surfaces. 
