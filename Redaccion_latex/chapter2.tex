
\newpage
\thispagestyle{empty}
\mbox{}
\chapter{Sistema de manera global}
\label{ch:chapter2}

Sea una función $f\left(x\right)$ con $x\in\mathbb{R}^{n}$. Además de ser continua y derivable para todo n. Aplicando el desarrollo en serie de Taylor siendo n = 1.

\begin{equation*}
	f\left(x\right)=f\left(a\right)+f^{'}\left(a\right)\left(x-a\right)+\frac{1}{2!}\cdot{f}^{''}\left(a\right){\left(x-a\right)}^{2}
\end{equation*}

Donde $f^{'}\left(a\right)$ y $f^{''}\left(a\right)$ se corresponden con la primera y segunda derivada de la función en torno a un punto cualquiera en el espacio "$a$", si en lugar de ello se hace con $x_*$, estando x lo suficientemente cerca de dicho punto.

\begin{equation*}
	f\left(x\right)=f\left(x_{*}\right)+f^{'}\left(x_{*}\right)\left(x-x_{*}\right)+\frac{1}{2!}\cdot{f}^{''}\left(x_{*}\right){\left(x-x_{*}\right)}^2 
\end{equation*}

Si $f^{'}\left(x_{*}\right)=0$ se tiene un máximo, mínimo o un punto de inflexión, teniendo esto en cuenta y despejando de la ecuación anterior.

\begin{equation*}
	f\left(x\right)-f\left(x_{*}\right)=\frac{1}{2!}\cdot{f}^{''}\left(x_{*}\right){\left(x-x_{*}\right)}^2 
\end{equation*}

\newpage

Se dan diferentes situaciones:

\begin{itemize}
	\item Si $f^{''}\left(x_{*}\right)<0\rightarrow{f}\left(x\right)-f\left(x_{*}\right)<0\rightarrow{f}\left(x\right)<f\left(x_{*}\right)\rightarrow{f}\left(x_{*}\right)$ es un máximo.
	\item Si $f^{''}\left(x_{*}\right)>0\rightarrow{f}\left(x\right)-f\left(x_{*}\right)>0\rightarrow{f}\left(x\right)>f\left(x_{*}\right)\rightarrow{f}\left(x_{*}\right)$ es un mínimo.
	\item Si $f^{''}\left(x_{*}\right)=0\rightarrow{f}\left(x\right)-f\left(x_{*}\right)=0\rightarrow{f}\left(x\right)=f\left(x_{*}\right)\rightarrow{f}\left(x_{*}\right)$ es un punto de inflexión.
\end{itemize}

Si se hace el mismo desarrollo y se expande el dominio para $n\geq{2}$, se obtiene:

\begin{equation*}
	f\left(x\right)=f\left(x_{*}\right)+\mathrm{\nabla}{f}{\left(x_{*}\right)}^{T}\left(x-x_{*}\right)+\frac{1}{2!}\cdot{\left(x-x_{*}\right)}^{T}\cdot{H}\left({f}\left(x_{*}\right)\right) 		\cdot\left(x-x_{*}\right)
\end{equation*}

Donde:

\begin{equation*}
	\begin{aligned}
		\mathrm{\nabla}{f}=
	\begin{bmatrix}
		\frac{\partial{f}}{\partial{x}_1} \\
		\frac{\partial{f}}{\partial{x}_2}  \\
		\vdots \\
		\frac{\partial{f}}{\partial{x}_n}
	\end{bmatrix}
	\end{aligned}
	\qquad\text{y}\qquad
	\begin{aligned}
	{H}\left(f\right)=\mathrm{\nabla}^{2}{f}= 	
	\begin{bmatrix}
		\frac{\partial^{2}{f}}{\partial{x}_{1}^{2}} & \frac{\partial^{2}{f}}{\partial{x}_{1}\cdot\partial{x}_{2}} & \cdots & \frac{\partial^{2}{f}}{\partial{x}_{1}\cdot\partial{x}_{n}}\\
		\frac{\partial^{2}{f}}{\partial{x}_{2}\cdot\partial{x}_{1}} & \frac{\partial^{2}{f}}{\partial{x}_{2}^{2}} & \cdots & \frac{\partial^{2}{f}}{\partial{x}_{2}\cdot\partial{x}_{n}}\\
		\vdots & \vdots & \ddots & \vdots\\
		\frac{\partial^{2}{f}}{\partial{x}_{n}\cdot\partial{x}_{1}} & \frac{\partial^{2}{f}}{\partial{x}_{n}\cdot\partial{x}_{2}} & \cdots & \frac{\partial^{2}{f}}{\partial{x}_{n}^{2}}
	\end{bmatrix}
	\end{aligned}
\end{equation*}\\

En este caso lo que interesa es que el gradiente de la función sea 0, es decir, que "$\mathrm{\nabla}{f}{\left(x_{*}\right)}=0$". Por lo tanto, los casos particulares previamente descritos adoptan un significado similar. 

\begin{equation*}
	f\left(x\right)-f\left(x_{*}\right)=\frac{1}{2!}\cdot{H}\left(f\right)\cdot\left(x-x_{*}\right)^2 
\end{equation*}

\begin{itemize}
	\item Si ${H}\left(f\right)<0$ (definida negativa) $\rightarrow{f}\left(x\right)-f\left(x_{*}\right)<0\rightarrow{f}\left(x_{*}\right)$ es un máximo.
	\item Si ${H}\left(f\right)>0$ (definida positiva) $\rightarrow{f}\left(x\right)-f\left(x_{*}\right)>0\rightarrow{f}\left(x_{*}\right)$ es un mínimo.
	\item Si ${H}\left(f\right)=0$ es indefinida es un punto silla.
\end{itemize}

En caso de las funciones para dos o más dimensiones, la condición necesaria para ser optimo es estar semidefinido, es decir, si $\mathrm{\nabla}{f}{\left(x_{*}\right)}=0$ y ${H}\left(f\right)$ es semidefinida, se tiene:

\begin{itemize}
	\item Es máximo si esta semidefinida negativa $\rightarrow{y}^{T}\cdot{H}\left({f}\right)\cdot{y}\leq{0}$
	\item Es mínimo si esta semidefinida positiva $\rightarrow{y}^{T}\cdot{H}\left({f}\right)\cdot{y}\geq{0}$
\end{itemize}

\section{Algoritmo de estimación del gradiente}

Aca describir consensus, funcion necesaria, poner la ecuacion para hacer referencia y decir que hay que definir aca.

\subsection{Descripción general}

hola

\subsection{Algoritmos de consenso}

hola

\subsection{Función lipzchiana}

hola


\section{Algoritmo de control de formación circular}

\subsection{Formación de control}

El diseño de control de los sistemas de robots de enjambre es bastante complejo porque se deben considerar muchos aspectos, como el mantenimiento de la formación, la comunicación y la coordinación entre robots.

El control de la formación, que es uno de los temas más activamente estudiados dentro del ámbito de los sistemas de múltiples agentes, generalmente tiene como objetivo impulsar a múltiples agentes para lograr las restricciones prescritas en sus estados. Dependiendo de la capacidad de detección y la topología de interacción de los agentes

el control de la formación basado en el consenso

la caracterización de los esquemas de control de formación en términos de la capacidad de detección y la topología de interacción de los agentes porque creemos que ambos están vinculados a las características esenciales del control de formación de múltiples agentes.

La caracterización de los esquemas de control de la formación en términos de la capacidad de detección y la topología de interacción conduce naturalmente a la pregunta de qué variables son detectadas y qué variables son controladas activamente por los sistemas de múltiples agentes para lograr la formación deseada. Los tipos de variables detectadas especifican el requisito sobre la capacidad de detección de agentes individuales. Mientras tanto, los tipos de variables controladas están esencialmente conectados a la topología de interacción. Específicamente, si las posiciones de los agentes individuales se controlan activamente, los agentes pueden moverse a sus posiciones deseadas sin interactuar entre sí. En el caso de que las distancias entre agentes se controlen activamente, la formación de agentes puede tratarse como un cuerpo rígido. Luego, los agentes deben interactuar entre sí para mantener su formación como un cuerpo rígido. En resumen, los tipos de variables controladas especifican la mejor formación deseada posible que pueden lograr los agentes, lo que a su vez prescribe el requisito sobre la topología de interacción de los agentes.

\subsection{Descripción general del algoritmo}



\section{Acción conjunta de ambos algoritmos}


\subsection{Algoritmo de ascenso de gradiente}
hola

El diagrama de fluyo ira aca

