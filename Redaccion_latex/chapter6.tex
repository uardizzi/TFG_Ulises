\newpage
\thispagestyle{empty}
\mbox{}

\chapter{Parrafor en sucio}
\label{ch:chapter6}

Introducción:

En base a todo lo descrito anteriormente se va a desarrollar a lo largo de los siguientes capítulos un  sistema multiagente de tipo enjambre que sea capaz de estimar un gradiente a partir de 3 o más robots dispuestos en una formación circular y estos sean capaces de desplazarse a una zona de interés definido como un máximo de una determinada función.


Capitulo 2: Según lo dicho en la charla de diciembre y en el pdf del campus aca va todo lo explicativo, tipo el optimo de una funcion, las funciones, bases de algoritmos....

Acá habría que definir los consensus algorithms e introducciones generales a varios aspectos (no se si meter lo de abajo al inicio acá), también un diagrama de flujo del control+estimador (este capitulo y el siguiente osea el 2+3 me deberían llevar como 18-20 paginas, llevaría de momento 1 de indice + 3-4 introducción + 18-20 (en dos capítulos))


Definir breves conceptos, problemas en la coordinación, descripción de las curvas de nivel para usar la estimación de gradiente.

