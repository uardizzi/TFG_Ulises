%\newpage
%\thispagestyle{empty}
%\mbox{}

\chapter{Conclusiones y futuras investigaciones}
\label{ch:chapter4}

Al evaluar el comportamiento del sistema completo variando sus parámetros más relevantes y en dos situaciones completamente diferentes se pueden extraer las siguientes conclusiones:

En primer lugar, el número de iteraciones vistas en cada gráfica a lo largo del capítulo anterior no van a suponer nada sobre el sistema real. En éste la importancia recae en la capacidad de los sensores para tomar los datos de manera continua y así poder llegar al punto de máxima concentración de sustancias.

En cuanto a la operación conjunta de los tres algoritmos. Estos satisfacen los objetivos dispuestos al inicio de la memoria que consistía en determinar la zona de máxima concentración de sustancias en superficies marítimas. Sin embargo, dados los distintos resultados obtenidos se puede deducir que determinar el número de agentes $N$, el radio $D$ y hasta la ganancia $\epsilon$ correspondiente al avance del algoritmo de ascenso, se deben elegir con especial cuidado dado que una mala elección de cualquiera de estos tres puede conllevar a que el sistema se ralentice.

Evaluando individualmente el efecto que tiene tanto el número de agentes N como el radio D sobre el algoritmo de estimación de gradiente y sobre el algoritmo de coordinación, se tiene:

\begin{itemize}
	\item Un aumento del número de agentes para la estimación va a representar una ventaja al obtener un gradiente teóricamente más cercano al real. Sin embargo, para la coordinación se puede dar el caso de que lleve mucho más tiempo disponerse uniformemente en la formación, y no solo eso sino que será necesario un aumento del radio en caso de que los vehículos tengan un riesgo de colisión.
	\item En el capítulo de resultados se vio como la dependencia con el radio invertía al hacer uso del algoritmo de coordinación, es decir, será proporcional aumentando el error si se aumenta el radio cuando no se utiliza ningún algoritmo de coordinación, mientras que será inversamente proporcional, si aumenta el radio disminuye el error, en el sistema que abarca los tres algoritmos.
\end{itemize} 

Un aspecto de vital importancia es el propio avance del algoritmo definido por el ascenso de gradiente. En este se recogen los dos errores dados por el algoritmo de estimación y el asociado al ángulo entre vecinos adyacentes del algoritmo de control de formación, a su vez se le añade la propia elección del valor de la ganancia $\epsilon$. Por lo que si juntas una mala definición de parámetros con los errores acumulados se pueden describir espirales incluso más pronunciadas que las referidas en la figura \ref{Epsilon_Var}.

Tanto en \cite{Estimacion_Gradiente} como en \cite{Adicional_Estimacion_1} dan una forma de corregir el error dado por la mala elección de dicha ganancia $\epsilon$ y las limitaciones apreciadas del algoritmos de ascenso de gradiente. Esto es hacer uso del método de Newton-Raphson, en el que aprovechas el hessiano de la función para desplazarte sobre su derivada. No obstante, se debe de estimar tal como se hizo con el gradiente, pero para ello se ha de definir un vehículo adicional en el centro de la formación.

Finalmente, un reto que actualmente se plantea es un cálculo del gradiente centralizado, en donde un único agente, se desplaza en la trayectoria circular y recoge las medidas correspondientes a cada uno de sus vecinos. Existe una segunda alternativa, que sería emplear consenso con el objetivo de que el cálculo del gradiente esté distribuido entre los agentes.

\nocite{Git__todos}